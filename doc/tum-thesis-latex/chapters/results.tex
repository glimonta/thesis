\chapter{Results}\label{chapter:results}

In this chapter we will describe the results and conclusions from the testing process in our work.

\section{Tests}

In the previous chapter we described the different types of tests that compose the test suite for Chloe.
Our main goal with the testing process was to increase the trust in the translation process and test that this process does not change the semantics of the programs.
For this purpose we created two sets of tests, one with correct programs and the other one with incorrect ones, as well as example programs in the language.
The example programs are meant to show how the language works and how programs can be written in it.
Additionally, they were beneficial in the beginning of the testing process to intuitively verify that the generated program was behaving according to the semantics.
For the incorrect programs, the tests help verify that no code is being generated for programs that are invalid in our semantics.

We generate code for correct programs and this code can be generated with tests.
The tests we generate will verify that the final states of both the execution in the semantics and the one in the machine are the same.
We define this as having the same content in every global variable and in the part of the dynamic memory that is reachable.
The specifics of this equivalence between states was detailed in the previous chapter.
The complete battery of tests can be found in the source code submitted with this work.

\section{Results}

Having a test suite that enabled us to check whether our translation process was being done correctly was remarkably valuable during the execution of this work.
Originally, we had simple programs written in the language which helped us to intuitively verify that the translation process was done correctly and the semantics of the program was not changed.
Subsequently, we were required to automate the testing process and design more specific tests.
We proceeded to add the test harness and to create the battery of tests.
The results we obtained from the tests were positive.
All the incorrect test cases failed as expected.
For all the correct cases we generate code with tests and all the tests generated for these programs run successfully.

It is important to note that the test cases were written by us, so we cannot state exact metrics about the results we obtained during the testing process.
It is possible that some cases are not completely covered by our test suite since the tests were written by us.
This is why new tests can, and should, be added to the test suite in order to keep increasing the trust in the translation process.
New tests should also be added as new functionality is added to the semantics.

As a recommendation for future work we would propose to translate to Chloe a set of tests developed by people outside of this work in order to have more precise metrics about the testing process.
An example of this would be to take a test suite made for C compilers such as gcc's C testsuites~\parencite{gcc-tests} and narrow the tests to ones that can be translated to our semantics, translate them from C to Chloe and generate the code with tests from them.
By doing so it would be possible to provide more precise metrics e.g.\ how much of the test suite is successfully covered by our work.
Unfortunately, this lies beyond the scope of this work.