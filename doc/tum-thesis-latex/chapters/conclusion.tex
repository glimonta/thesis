\chapter{Conclusion and Future Work}\label{chapter:conclusion}

\section{Conclusions}

In this work we have managed to successfully formalize the semantics for an imperative language called Chloe which covers a subset of the C language.
Chloe has the following features: variables, arrays, pointer arithmetic, while loop construct, if-then-else conditional construct and dynamic memory.
We have formalized a small-step semantics in the Isabelle/HOL theorem prover for Chloe and proven the semantics to be deterministic.
We also present an interpreter for the language, thus allowing for programs written in Chloe to be executed within the Isabelle/HOL environment.
In order to do so we needed the determinism proofs for the semantics.

Another result of this work was to present a code generator for the Chloe language.
This code generator will translate programs from the formal semantics to real C code which can be executed.
We guarantee that the generated program can be compiled and executed in a machine given that it is a program without undefined behavior, which complies with the restrictions assumed in the formalization of our semantics (section~\ref{subsection:restrictions_commands}) and doesn't run out of memory during execution.
Moreover, we will only generate C code for programs that are valid and execute correctly in our interpreter.

We faced some problems when formalizing the semantics such as allowing only programs with defined behavior to be executed in our semantics.
This means we had to detect undefined behavior according to the C standard~\parencite{c99} such as integer overflow and consider it an error in our semantics.
We also faced a problem when formalizing the semantics for the memory allocation function because we assume an unlimited amount of memory, whereas the resources in a machine are limited.
In order to solve this problem we decided to change the way a memory allocation call would be translated to C code.
We wrap C's malloc function in a function defined by us that would abort the program if a call to malloc fails.
We also had to assume architecture restrictions over the target machine where the generated code will be executed and generate code that checks if the machine satisfies our assumptions.

Finally, we also present a test harness and a test suite for testing the translation process.
The goal of this test suite created is to increase the trust in the translation process and to make sure that the semantics of a Chloe program does not change when translated to C code.
In this test suite we include incorrect cases that cover all the border cases or cases where no code should be generated due to the program facing an error.
We also include example programs to demonstrate how the language works and how the programs are translated to C code.
There is also a set of correct tests written that are meant to generate code.

To guarantee that the semantics of a program is not changed by the translation process we also include a way of generating code with tests.
These tests are meant to check that the final state of the program executed in the interpreters within Isabelle/HOL is the same as the state the program is in after being compiled and executed.
Two states are equivalent when the global variables and the reachable dynamic memory, at the end of execution, have the same content.
For this purpose we wrote a test harness in Isabelle which translates the tests to be made into a set of C macros that we define outside of Isabelle/HOL, and include in our compilation process, which test the equivalence between final states.
In order to not run these tests manually we defined a little bash script which runs all the existing tests automatically.

\section{Future work}

In this section we will point some directions in which this work can be taken in the future.
The time frame available for doing this work made it impossible to include these features in the scope of our work.

First of all, we can upgrade the testing process by parsing C tests from an outer test suite (as mentioned in the previous chapter) and translating them to Chloe.
After having the translation to Chloe we can generate code and tests for them in order to enhance the current test suite we have provided with this work.

Another interesting direction this work might take is to formalize an axiomatic semantics in order to reason about programs and their properties in Chloe.
Since Chloe has pointers as a feature it would be necessary to formalize a separation logic~\parencite{sep-logic} in order to reason about pointers in programs.
By extending the work in this direction it would be possible to show partial and total correctness properties proofs of our programs.

One of the features not included in the scope of our work is a proven sound and correct static type system.
This would allow to reason about type safety for programs written in Chloe.

There is an Isabelle Refinement Framework~\parencite{Refine_Monadic-AFP}, which provides a way of formulating non-deterministic algorithms in a monadic style and refine them to obtain an executable algorithm.
It also provides tools to reason about these programs.
Another source of future work would be to link our language to the Isabelle Refinement Framework, so that programs from the framework can be refined to programs in Chloe.

Finally, the set of features that are currently supported by Chloe is limited.
Another way to improve the results from this work is to expand the set of features supported by Chloe.
This might include expanding the set of expressions and instructions, adding I/O operations or support for concurrency.
