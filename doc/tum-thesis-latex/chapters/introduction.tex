\chapter{Introduction}\label{chapter:introduction}

In this chapter we wil firstly discuss the motivation behind this work.
Followed by theoretical background on concepts relating to semantics and Isabelle/HOL.

Then we will describe the features of the language used in this work.

Finally we will proceed to outline the contents of the rest of this work.

\section{Motivation}

Produce reliable software.

C standard allows for ambigueties, formal semantics as an alternative to this.



\section{Theoretical Background}

\subsection{Semantics of a programming language}

In this section we'll give a short introduction to the concepts regarding semantics we want the reader to be familiar with beforehand.

\subsubsection{Definition}

The semantics of a programming language is the intended meaning of constructs in that language.
According to Tenent~\parencite{tennent} in order to support and define these meanings for a construct in a programming language a rigorous mathematical theory of the semantics of programming languages is needed.

As noted by Nielson and Nielson~\parencite{nielson}, the rigorous nature of this study is due to the fact that it can reveal ambiguities or underlying complexities in documents defined used natural language, and also that this mathematical rigour is necessary for correctness proofs.
Many big programming languages, e.g.\ the C language, have their reference document, where the semantics of the language is explained, written in natural language.
Sadly natural language leads to subtle ambiguities that further make it harder to reason about programs in this language.

The lack of a rigorous mathematically defined semantics also makes it harder for developers of tools for that programing language, i.e.\ compilers, to write accurate tools for it since, due to the ambiguities, some part of the language is subject to the interpretation of the reader.
This doesn't happen with formally defined terms, by using these we lose the possibility of ambiguity and can ensure that the meaning of a construct in the language always has the same meaning as defined in its semantics.

In order to further clarify the definition and the different types of semantics there exists we are going to use an example we considered relevant from Nielson and Nielson~\parencite{nielson}:

If we take the following construct
\begin{equation*}
z:=x; x:=y; y:=z
\end{equation*}
where `$:=$' is assignment to a variable and `$;$' is the instruction sequentiation.
Syntactically this is three statements separated by `$;$', where each statement is confomed by a variable, the symbol `$:=$' and a second variable.

The \textit{semantics} of this construct is the meaning of this program.
Semantically, this program, exchanges the values of $x$ and $y$ (using $z$ as a temporary variable).

\subsubsection{Types of semantics}

The previous was a rough explanation of the semantics of that particular program.
This semantical explanation can clearly be done in a more detailed way where we exaplain the meaning of the sequence of statements separated by `$;$' and the one of the individual assign statement.

There are many differnt approaches on how to formalize the semantics of a programming language.
Each one of them has a different idea behind it and have a different degree of usefulness depending on the purpose one has for the semantics.
The most widely used approaches are:

\begin{comment}
  \begin{itemize}
    \item{Operational semantics:}
    The meaning is specified by the computation on a machine when executing the program.
    The interest lies on the ``\textit{how} the effect of a computation is produced.''~\parencite{nielson}
    \item{Denotational semantics:}
    The meaning of a construct are modelled mathematically by objects that represent the effect of executing the construct.
    Here the main interest lies in the effect and not how it was obtained.
    \item{Axiomatic semantics:}
    Here the main interest lies in expressing some properties of the construct in the form of assertions.
  \end{itemize}
\end{comment}

\subsubsection{Operational semantics}

A semantics is defined using the operational approach when the focus is turned to \textit{how} a construct is executed, it can be thought of as an abstraction of the execution of the program in a machine.~\parencite{nielson}

There's two different kinds of operational semantics, namely \textit{small-step semantics} (or structural operational semantics) and \textit{big-step semantics} (or natural semantics).

The operational explanation of the meaning of this program indicates how it is executed.

Our example consists of a series of three statements separated by `$;$' and each of this statements is an assignment.
The operational semantics will define how are sequence of statements and assignment statement executed.

For this particular construct the semantics will be as follows:

\begin{itemize}
\item{To execute a sequence of statements, each statement is executed in a left-to-right order.}
\item{To execute an assignment statement between two variables, the value of the right-hand-side variables is determined and assigned to the left-hand-side variable.}
\end{itemize}


We will proceed to define the two different kind of operational semantics by using the example introduced in the previous section and stating the different semantics for that particular construct.

\paragraph{Big-step semantics}

This kind of semantics represents the execution of a program from an initial state to a final state in one single step, therefore it does not allow for explicit inspection of intermediate execution states.~\parencite{nipkow}

The execution of the whole program will look as follows:

\begin{equation*}
\langle z:=x; x:=y; y:=z, s_{0} \rangle \rightarrow s_{3}
\end{equation*}

where the following abbreviations were used:
\begin{align*}
s_{0} &= [x\mapsto5, y\mapsto7, z\mapsto0]\\
s_{3} &= [x\mapsto7, y\mapsto5, z\mapsto5]
\end{align*}

However we can obtain the following \enquote{derivation sequence} for the previous construct:

\begin{equation*}
\cfrac{
  \cfrac{\langle z:=x, s_{0}\rangle \rightarrow s_{1} \qquad \langle x:=y, s_{1} \rangle \rightarrow s_{2}}
    {\langle z:=x; x:=y, s_{0} \rangle \rightarrow s_{2}}
  \qquad
  \langle y:=z, s_{2} \rangle \rightarrow s_{3}
  }
  {\langle z:=x; x:=y; y:=z, s_{0} \rangle \rightarrow s_{3}}
\end{equation*}

where the following abbreviations were used:
\begin{align*}
s_{0} &= [x\mapsto5, y\mapsto7, z\mapsto0]\\
s_{1} &= [x\mapsto5, y\mapsto7, z\mapsto5]\\
s_{2} &= [x\mapsto7, y\mapsto7, z\mapsto5]\\
s_{3} &= [x\mapsto7, y\mapsto5, z\mapsto5]
\end{align*}

Executing $z:=x$ in the state $s_{0}$ will yield the state $s_{1}$ and executing $x:=y$ in the state $s_{1}$ will yield the state $s_{2}$.
Therefore executing $z:=x; x:=y$ in state $s_{0}$ will yield the state $s_{2}$.
Also, executing $y:=z$ in the state $s_{2}$ will yield the state $s_{3}$.
Finally, executing the whole program $z:=x; x:=y; y:=z$ in state $s_{0}$ will yield the state $s_{3}$.

\paragraph{Small-step semantics}

Sometimes we want to have a bit more of information regarding intermediate states, that's why small-step semantics exists.

This kind of semantics represents small, atomic execution steps in a program and allows for reasoning about how far a program has been executed and to explicitly inspect partial executions.~\parencite{nipkow}

In this example we will go from the complete construct and take small-steps (denoted by `$\Rightarrow$') that yields the remaining of the construct after executing a step and the state resulting from it, until the whole construct is executed.

Supposing we have a state where the variable $x$ has the value $5$, the variable $y$ has the value $7$ and the variable $z$ the value $0$ and the construct from our example we would obtain the following \enquote{derivation sequence}:

\begin{equation*}
\begin{split}
& \phantom{\Rightarrow} \phantom{=} \langle z:=x; x:=y; y:=z, [x\mapsto5, y\mapsto7, z\mapsto0]\rangle\\
& \Rightarrow \phantom{=} \phantom{z:=x} \langle x:=y; y:=z [x\mapsto5, y\mapsto7, z\mapsto5]\rangle\\
& \Rightarrow \phantom{=} \phantom{z:=x; x:=y} \langle y:=z [x\mapsto7, y\mapsto7, z\mapsto5]\rangle\\
& \Rightarrow \phantom{=} \phantom{z:=x; x:=y; y:=z} [x\mapsto7, y\mapsto5, z\mapsto5]
\end{split}
\end{equation*}

What happened here is that in the first step the statement $z:=x$ is executed and the value of the variable $z$ changes to $5$, $x$ and $y$ remain unchanged.
After executing the first statement the program we're left with is $x:=y; y:=z$.
We execute the second statement $x:=y$ and the value of $x$ changes to $7$, $y$ and $z$ remain unchanged.
We're left then with the program $y:=z$, after executing this final statement the value of $y$ changes to $5$, $x$ and $z$ remain unchanged.

Then we have that the behaviour of this program was to change the values of $x$ and $y$ using $z$ as a temporary variable.


\subsubsection{Denotational semantics}
In the denotational semantics we stop focusing on \textit{how} a construct is executed and redirect our focus to the \textit{effect} of executing the construct.
This approach assists us by giving a \textit{meaning} to the constructs in a language.~\parencite{nipkow}
We model this approach by mathematical functions, we will have a function for each construct that defines the meaning of it and these functions operate over states, taking the initial state and yielding the state resulting from applying the effect of the construct.

Taking the previous example we can define the effects of the different constructs we have: sequencing and assignment statements.

\begin{itemize}
\item{The effect of functional composition of each individual statement will define the effect of a sequence of statements.}
\item{The effect of assignment between two variables is defined by a function that takes a state and yields the same state where the current value of the right-hand-side variable is updated with the new value of the left-hand-side variable.}
\end{itemize}

For this particular example construct we would obtain functions of the form $S [\![ z:=x ]\!]$, $S [\![ x:=y ]\!]$ and $S [\![ y:=z ]\!]$ for each individual statement.
On the other hand for the complete compound statement that is the whole program we will get the following function:

\begin{equation*}
S [\![ z:=x; x:=y; y:=z ]\!] = S [\![ y:=z ]\!] \circ S [\![ x:=y ]\!] \circ S [\![ z:=x ]\!]
\end{equation*}

Executing the complete program $z:=x; x:=y, y:=z$ on a particular step would have the effect of \textit{applying} the function $S [\![ z:=x; x:=y; y:=z ]\!]$ to the initial state $[x\mapsto5, y\mapsto7, z\mapsto0]$:

\begin{align*}
S [\![ z:=x; & x:=y; y:=z ]\!]([x\mapsto5, y\mapsto7, z\mapsto0])\\
&= (S [\![ y:=z ]\!] \circ S [\![ x:=y ]\!] \circ S [\![ z:=x ]\!])([x\mapsto5, y\mapsto7, z\mapsto0])\\
&= S [\![ y:=z ]\!](S [\![ x:=y ]\!] (S [\![ z:=x ]\!]))([x\mapsto5, y\mapsto7, z\mapsto0])\\
&= S [\![ x:=y ]\!] (S [\![ z:=x ]\!])([x\mapsto5, y\mapsto7, z\mapsto5])\\
&= S [\![ z:=x ]\!]([x\mapsto7, y\mapsto7, z\mapsto5])\\
&= [x\mapsto7, y\mapsto5, z\mapsto5]
\end{align*}

We focus on the resulting state that represents the effect the program had in the initial state.
It is easier to reason about programs using this approach since it is similar to reasoning about mathematical objects.
Although it is important to note that establishing a firm mathematical basis to do so is not a trivial task.
The denotational approach is useful since it can easily be adapted to represent some properties of programs.
Examples of this are variable initialization, constant folding and reachability.
This approach is used when the priority is to reason about a program and its properties.~\parencite{nielson}

\begin{comment}
what the program does oposed to how it does it. Modifying mathematical objects.
\end{comment}

\subsubsection{Axiomatic semantics}

This final approach is used when one is interested in \textit{partial correctness properties} of programs.
Partial correctness with respect to a construct, a precondition and a postcondition whenever the following implication holds:

\begin{displayquote}
If the precondition holds before the construct is executed and the execution of the construct terminates then the postcondition holds for the final state.
\end{displayquote}

It is partial correctness because it depends on the termination of the contruct's execution.~\parencite{nipkow}


The following partial corretness property is defined for the construct of our example:

\begin{equation*}
\lbrace x=n \land y=m \rbrace z:=x; x:=y; y:=z \lbrace x=m \land y=n \rbrace
\end{equation*}

where $\lbrace x=n \land y=m \rbrace $ and $\lbrace x=m \land y=n \rbrace $ are the precondition and postcondition, respectively.
$n$ and $m$ indicate the initial values of $x$ and $y$.
The state $[x\mapsto5, y\mapsto7, z\mapsto0]$ fulfils the precondition if $n=5$ and $m=7$ are taken.
After the partial correcteness property is \textit{proved} then it can be deduced that \textit{if} the program terminates \textit{then} it will do so in a state where $x$ is $7$ and $y$ is $5$.
Even though it is clear on this example program that it terminates it is important to note that, as was stated previously, the partial correctness property doesn't ensure a program's termination and is dependent on it.

This approach relies on a \textit{proof system} or inference rules for deriving and proving partial correctness properties.~\parencite{nipkow}
The following ``proof tree'' can express a proof for the above partial corretness property:

\begin{equation*}
\cfrac{
  \cfrac{ \lbrace p_{0} \rbrace z:=x \lbrace p_{1} \rbrace \qquad \lbrace p_{1} \rbrace x:=y \lbrace p_{2} \rbrace }
    { \lbrace p_{0} \rbrace z:=x; x:=y \lbrace p_{2} \rbrace }
  \qquad
 \lbrace p_{2} \rbrace y:=z \lbrace p_{3} \rbrace
  }
  { \lbrace p_{0} \rbrace z:=x; x:=y; y:=z \lbrace p_{3} \rbrace }
\end{equation*}

where the following abbreviations were used:
\begin{align*}
p_{0} &= x=n \land y=m\\
p_{1} &= y=m \land z=n\\
p_{2} &= x=m \land z=n\\
p_{3} &= x=m \land y=n
\end{align*}

The benefit of using this approach is that we are provided with an easy way to prove program properties by the proof system.


\subsection{Isabelle/HOL}

\begin{comment}
We will shortly talk about what Isabelle/HOL is and then refer the reader to the book by Nipkow and Klein.~\parencite{nipkow}
\end{comment}

Reasoning about a formal semantics using just pen and paper can very quickly become an error-prone and complicated process.
This is why we rely on a theorem prover, namely Isabelle, where we can write definitions and then prove theorems in a sound manner.

The semantics for the language used in this work is formalized with the assistance of Isabelle/HOL.
Isabelle/HOL is the specialization of Isabelle for HOL (Higher Order Logic). ~\parencite{nipkow}

The definitions for the formal semantics and proofs that accompany it are written in this Isabelle/HOL.
We won't specify the details from Isabelle/HOL in this section but rather refer the reader to \textit{Concrete Semantics with Isabelle/HOL}~\parencite{nipkow} which shows an introduction to Isabelle/HOL and theorem proving.

\subsection{Language used in this work}

In this work we formalize the semantics for a small programming language called Chloe (TODO).
It is a subset of the C programming language.
The syntax and semantics of this programming language are discussed in further detail later in chapter 3.

This language has the following features: variables, arrays, pointer arithmetic, while loop construct, if-then-else conditional construct and dynamic memory.
Sadly, the scope of this project was limited to the features mentioned previously and there are several features that are currently not supported by the language.
These features are a sound and correct type system, concurrency, I/O operations, goto, labels, break and continue.

While it will be convenient in the future to have the features not covered by the scope of this work, the current set of features supported by Chloe (TODO) is enough to show some relevant program examples and it has enough expressive power to be Turing-complete.


\begin{comment}
The language should probably have a name.
I have to give one to it.
Here I write what kind of features my language has.
Variables, loops, functions, pointers, etc.
And the features I do not support.
Concurrency, types, etc.
\end{comment}

\section{Thesis Structure}

Chapter 2: Related work.

Chapter 3: Syntax and semantics.

Chapter 4: Pretty printing.

Chapter 5: Testing.

Chapter 6: Results.

Chapter 7: Conclusion and future work.
