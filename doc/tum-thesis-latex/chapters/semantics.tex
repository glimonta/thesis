\chapter{Syntax and Semantics}\label{chapter:semantics}

\section{Isabelle}\label{section:isabelle}

Why is Isabelle chosen to use in this work instead of some other tool.
IsabelleHOL, Isabelle's code generation.
Maybe this should be moved to the theoretical background.

\section{Expressions}\label{section:expressions}

Our imperative language is called Chloe and it represents a subset of the C language.
We will now proceed to discuss more in detail the syntax and semantics of the expressions in Chloe.

\subsection{Syntax}\label{subsection:syntax_expressions}

Here we proceed to describe the \textbf{abstract syntax} for the expressions of the Chloe language.

\begin{figure}
  \begin{lstlisting}[frame=single]
  type_synonym vname = string

  datatype exp = Const int
               | Null
               | V      vname
               | Plus  exp exp
               | Subst exp exp
               | Minus exp
               | Div   exp exp
               | Mod   exp exp
               | Mult  exp exp
               | Less  exp exp
               | Not   exp
               | And   exp exp
               | Or    exp exp
               | Eq    exp exp
               | New   exp
               | Deref exp
               | Ref   lexp
               | Index exp exp
  and
  datatype lexp = Deref exp
                | Indexl exp exp
  \end{lstlisting}

  \caption{Chloe expressions}
  \label{fig:chloe_expressions}
\end{figure}

In figure~\ref{fig:chloe_expressions} we can find the datatype created in Isabelle for the expressions, where \verb|int| is the predefined type for integers and \verb|vname| stands for variable name.

We define two new datatypes, one for expressions and one for left-hand-side expressions.
It is important to differentiate between these two in the case where we are dealing with pointer expressions.
For instance, let's take the following C expressions:

\begin{lstlisting}[language=C, frame=single]
foo = *bar;
*baz = 1;
\end{lstlisting}

In the first expression \verb|*bar| is on the right-hand-side, in this case we want \verb|*bar| to yield a value we can then assign to \verb|foo|.
On the other hand, in the second expression \verb|*baz| is on the left-hand-side, in this case we want \verb|*baz| to yield an address to which we can assign the value \verb|1|.

This also occurs with array indexing.
In order to correctly model the semantics for the Chloe expressions it is necessary to have this distinction between left-hand-side expressions and right-hand-side expressions.
In future sections we will proceed to refer to right-hand side expressions as simply expressions and we will use LHS instead of left-hand-side when referring to left-hand-side expressions.

Chloe supports constant expressions, null pointer expression, variables as well as the following operation over expressions: addition, substraction, unary minus, division, modulo, multiplication, less than, not, and, or, equality.
We also have a \verb|New| expression which is the equivalent of a \verb|malloc| call in C.
We have dereferencing, referencing and array subscripting, these are in C the \verb|*|, \verb|&| and \verb|[]| operators, respectively.
Finally as LHS expressions we have dereferencing and array subscripting.

\subsubsection{Types}

In the Chloe language we have two types, namely integers and addresses.
We differentiate between values of type integer and addresses in order to correctly define our semantics.
Next, we will proceed to present the details of the two types in the Chloe language.

\paragraph{Integers}

We define the following type synonyms in Isabelle:

\begin{lstlisting}[frame=single]
type_synonym int_width = 64
type_synonym int_val = int_width word
\end{lstlisting}


The term \verb|int_width| refers to the width of the integer value.
In this case we assume a value of 64 since we are working with a 64 bit architecture.
This parameter indicates the semantics to assume we are working with a 64 bit architecture where the upper and lower bounds for an integer are defined in figure~\ref{fig:int_bounds}.

\begin{figure}
  \begin{lstlisting}[frame=single, mathescape=true]
  abbreviation INT_MIN :: int where INT_MIN $\equiv$ - (2^(int_width - 1))
  abbreviation INT_MAX :: int where INT_MAX $\equiv$  ((2^(int_width - 1)) - 1)
  \end{lstlisting}

  \caption{Integer lower and upper bounds}
  \label{fig:int_bounds}
\end{figure}

When working with a different architecture this parameter can be changed in order to comply with the architechture's requierements.

Also our integers are defined as words of length \verb|int_width| (in this case \verb|64|).
Since we are not using the Isabelle's predefined \verb|int| type, in order to work with words and to support code generation for them we use the Native Word entry in the Archive of Formal Proofs.~\parencite{Native_Word-AFP}.

From now on we will mostly refer to the 64 length words we use to represent integers in our language just as integers.
Note that, unless explicitly stated so in the text, for simplicity we will use the word `integer' to refer to a 64 length word instead of Isabelle's predefined \verb|int| type.

\paragraph{Addresses}

We define the following \verb|datatype| in Isabelle to represent addresses:

\begin{lstlisting}[frame=single, mathescape=true]
datatype addr = nat $\times$ int
\end{lstlisting}

An address will then be a pair conformed by a natural number and an integer (which is an Isabelle predefined \verb|int|), these represent a \verb|(block_id, offset)| pair.
In the following section we will proceed to explain the layout of the memory.

\subsubsection{Values} The values for any expression are defined as follows:

\begin{lstlisting}[frame=single, mathescape=true]
datatype val = NullVal | I $int\_val$ | A $addr$
\end{lstlisting}

where \verb|NullVal| corresponds to the null pointer expression, \verb|I| $int\_val$ corresponds to an integer value, and \verb|A| $addr$ corresponds to an address value.
When evaluating an expression we can obtain any of these three values.

\subsubsection{Memory}

We model dynamic memory in the following way:

\begin{lstlisting}[frame=single, mathescape=true]
type_synonym mem = val option list option list
\end{lstlisting}

The memory is represented by a list of allocated blocks, and each of these blocks is conformed by a list of cells with the values in memory.
For every block there's two possibilities: an allocated block or an unallocated block, this is modeled by the use of the \verb|option| type, where \verb|Some| $l$ (where $l$ is of type \verb|val option list|) denotes an allocated block and \verb|None| an unallocated one.
Every block is conformed by a list of cells that contain the values in memory.
Each cell can have different values depending on whether it is uninitialized or it holds a value.
An uninitialized cell in memory is represented by the value \verb|None|.
Whereas a cell holding a value is represented by the value \verb|Some| $v$, where $v$ is of type \verb|val|.

There are four main operations that can manage the memory, these are \verb|new_block|, \verb|free|, \verb|load| and \verb|store| and these are specified in figure ~\ref{fig:mem_operations}.
Each of these operations can fail, therefore their return type is $\tau\ \verb|option|$.
The values of that type are either \verb|None| when the operation fails or $\verb|Some|(v)$ where $v$ is of type $\tau$.

The functionalities of the memory management operations are the following:

\begin{itemize}
  \item{\verb|new_block| is the function that allocates a new block of dynamic memory of a given size.
  This function will fail in the case a size less or equal than zero is given, it can also fail if a value of a different type than an integer is given.
  Upon successful execution the function will yield the beginning address of the new block along with the modified memory.}
  \item{\verb|free| is the function that deallocates a block from the dynamic memory.
  This function will fail in the case where the given address is not a valid one in memory.
  Upon successful execution the function will yield a new state that includes the updated memory.}
  \item{\verb|load| is the function that given an address retrieves a value stored in the memory cell denoted by the given address.
  This function will fail in the case where the given address is not a valid one in memory.
  Upon successful execution the function will yield the value stored in memory.}
  \item{\verb|store| is the function that given an address stores a value in the memory cell denoted by the given address.
  This function will fail in the case where the given address is not a valid one in memory.
  Upon successful execution the function will yield a new state that includes the updated memory.}
\end{itemize}

\begin{figure}
  \begin{lstlisting}[mathescape=true, frame=single]
  new_block :: val $\Rightarrow$ mem $\Rightarrow$ (val $\times$ mem) option
  free      :: addr $\Rightarrow$ val  $\Rightarrow$ visible_state $\Rightarrow$ visible_state option
  load      :: addr $\Rightarrow$ mem $\Rightarrow$ val option
  store     :: addr $\Rightarrow$ val  $\Rightarrow$ visible_state $\Rightarrow$ visible_state option
  \end{lstlisting}

  \caption{Memory Management operations}
  \label{fig:mem_operations}
\end{figure}

It is important to note that the only reasons why the memory allocation can fail are the ones described above.
Since we assume the memory to be unlimited, there will not be a case where a \verb|new| call fails due to a lack of memory.


Also one of the types used in figure ~\ref{fig:mem_operations}, namely \verb|visible_state|, we have not yet talked about, it will be discussed in further detail in section~\ref{paragraph:visible_state}.

This memory model is inspired in the work of Blazy and Leroy~\parencite{compcert}, it is a simpler model ajusted to satisfy our needs.


\subsection{Semantics}\label{subsection:semantics_expressions}

The semantics of an expression is its value.
For expressions such as $2 + 2$ is trivial to know the value is $4$.
On the other hand when we have expressions with variables, such as $foo + 42$, then we depend on the value of the variable.
Therefore we need to know the values of a variable, these values are stored in the program state.

The program state is a bit more complicated than what we are about to present.
Section~\ref{section:states} is dedicated exclusively to discuss states but we will proceed to clarify in this section the part of the state needed for the expressions' semantics.

\paragraph{Valuations}\label{paragraph:valuation}

We define the type for a valuation as follows:

\begin{lstlisting}[frame=single, mathescape=true]
type_synonym valuation = vname $\Rightarrow$ val option option
\end{lstlisting}

A valuation is a function that maps a variable name to a value.
The return type is \verb|val option option| to model the following states of a variable's value: undefined, uninitialized and holding a value.
Therefore, given a variable name, this function can yield one of the following three results:

\begin{itemize}
  \item{\verb|None|, which represents a variable that is undefined.}
  \item{\verb|Some None|, which represents a variable that is defined but uninitialized.}
  \item{\verb|Some| $v$, which represents a defined initialized variable that holds the value $v$.}
\end{itemize}

\paragraph{Visible States}\label{paragraph:visible_state}

A visible state is the state that is visible at any point of a program that is not a function call or a function return.
The defined type for it is as follows:

\begin{lstlisting}[frame=single, mathescape=true]
type_synonym visible_state = valuation $\times$ valuation $\times$ mem
\end{lstlisting}

A visible state is a tuple conformed by a valuation function for the local variables, a valuation function for the global variables and the dynamic memory of the program.


Now we can proceed to introduce the semantics for the expressions in Chloe.
As we said before the semantics of an expression is its value, therefore we have defined two evaluation functions, one that computes the value of an expression and one that computes the value of a LHS expression.
These functions defined as follows:

\begin{lstlisting}[frame=single, mathescape=true]
eval   :: exp $\Rightarrow$ visible_state $\Rightarrow$ (val $\times$ visible_state) option
eval_l :: lexp $\Rightarrow$ visible_state $\Rightarrow$ (addr $\times$ visible_state) option
\end{lstlisting}

Where \verb|eval|, given an expression and a visible state, will yield the value of that expression and the visible state resulting after evaluation of that expression.
\verb|eval_l|, given a LHS expression and a visible state, will yield the value of that expression (which must be an address) and the state resulting after the evaluation of the LHS expression.
Notice that the resulting type for these evaluation functions are an \verb|option| type.
This is because these functions can fail, a failure can happen anywhere in the evaluation of the expression and if a failure is found then it will propagate until the whole expression evaluation returns a \verb|None| value indicating an error in evaluation.
Expression evaluation might fail for several reasons which include, but are not limited to, variable undefinedness, an operation failing because it has some illegal operand, trying to access an invalid part of the memory, overflow, among others.
Therefore if there's an error early in the expression evaluation semantics, it will be detected and propagated as a \verb|None| value which indicates an error state.


The \verb|eval| and \verb|eval_l| functions depend on several other defined functions in order to properly compute the values of expressions.
A definition of all the auxiliary functions for \verb|eval| and \verb|eval_l| is given in figure~\ref{fig:aux_fun_eval}.
Except for \verb|div_towards_zero| and \verb|mod_towards_zero|, each of these operations can fail, therefore their return type is $\tau$ \verb|option|.
The values of that type are either \verb|None| when the operation fails or $\verb|Some|(v)$ where $v$ is of type $\tau$.

\begin{figure}
  \begin{lstlisting}[mathescape=true, frame=single]
  detect_overflow  :: int $\Rightarrow$ val option
  read_var         :: vname $\Rightarrow$ visible_state $\Rightarrow$ val option
  plus_val         :: val $\Rightarrow$ val $\Rightarrow$ val option
  subst_val        :: val $\Rightarrow$ val $\Rightarrow$ val option
  minus_val        :: val $\Rightarrow$ val option
  div_towards_zero :: int $\Rightarrow$ int $\Rightarrow$ int
  div_val          :: val $\Rightarrow$ val $\Rightarrow$ val option
  mod_towards_zero :: int $\Rightarrow$ int $\Rightarrow$ int
  mod_val          :: val $\Rightarrow$ val $\Rightarrow$ val option
  mult_val         :: val $\Rightarrow$ val $\Rightarrow$ val option
  less_val         :: val $\Rightarrow$ val $\Rightarrow$ val option
  not_val          :: val $\Rightarrow$ val option
  to_bool          :: val $\Rightarrow$ bool option
  eq_val           :: val $\Rightarrow$ val $\Rightarrow$ val option
  new_block        :: val $\Rightarrow$ mem $\Rightarrow$ (val $\times$ mem) option
  load             :: addr $\Rightarrow$ mem $\Rightarrow$ val option
  \end{lstlisting}

  \caption{Auxiliary functions for eval and eval\_l}
  \label{fig:aux_fun_eval}
\end{figure}

The functionalities of the auxiliary functions are the following:

\verb|detect_overflow| is the function that detects integer overflow.
It takes an Isabelle predefined integer value as a parameter and checks for overflow with the bounds described in figure~\ref{fig:int_bounds}.
This function will fail whenever overflow is detected.
Upon sucessful execution the function will yield the value correspoding to the given integer parameter.

\verb|read_var| is the function that computes the value of a variable.
This function will fail whenever the variable name given as a parameter corresponds to an undefined variable.
Upon successful execution the function will yield the value of the variable.
In order to compute the value of said variable this function checks the local valuation in the visible state and proceeds to yield the value of the variable if it is defined there.
In the case where the variable is not defined in the local scope, the function will proceed to check the global scope and yield the value of the variable.

\verb|plus_val| is the function that computes the value of an addition between two values.
This function will fail whenever overflow is detected or when anything different than two integers or an address and an integer (in that specific order) are given as parameters to the function.
Upon successful execution with two integer values the function will yield an integer value corresponding to the addition of those operands.
Upon successful execution with an address and an integer the function will yield an address value corresponding to adding the integer offset to the original address value.

\verb|subst_val| is the function that computes the value of a substraction between two values.
This function will fail whenever overflow is detected or when anything different than two integers or an address and an integer (in that specific order) are given as parameters to the function.
Upon successful execution with two integer values the function will yield an integer value corresponding to the substraction of those operands.
Upon successful execution with an address and an integer the function will yield an address value corresponding to substracting the integer offset to the original address value.

\verb|minus_val| is the function that computes the value of the unary minus operation over a value.
This function will fail whenever overflow is detected or a value different from an integer is given as a parameter.
Upon successful execution the function will yield an integer value corresponding to the result of negating the value given as a parameter.

\verb|div_towards_zero| is the function that performs integer division with truncation towards zero.

\verb|div_val| is the function that computes the value of division between two values.
This function will fail whenever either overflow or division by zero are detected or when the parameters given to the function are different from integers.
Upon successful execution the function will yield an integer value corresponding to the result of performing integer division on the function operands.

\verb|mod_towards_zero| is the function that performs the modulo operation with truncation towards zero.

\verb|mod_val| is the function that computes the value of performing the modulo operation between two values.
This function will fail whenever either overflow or modulo by zero are detected or when the parameters given to the function are different from integers.
Upon successful execution the function will yield an integer value corresponding to the result of performing integer modulo on the function operands.

\verb|mult_val| is the function that computes the value of a multiplication between two values.
This function will fail whenever overflow is detected or when anything different from integers are given as parameters to the function.
Upon successful execution the function will yield an integer value corresponding to the multiplication of the function operands.

\verb|less_val| is the function that computes the value of performing the less than operation between two values.
This function will fail whenever anything different from integers are given as parameters to the function.
Upon successful execution the function will yield an integer value of \verb|I| $1$ when the first operand is smaller than the second one, and an integer value of \verb|I| $0$ otherwise.

\verb|not_val| is the function that computes the value of performing logical negation over a value.
This function will fail whenever a parameter different from an integer is given.
Upon successful execution the function will yield an integer value of \verb|I| $1$ when the operand is an integer of value \verb|I| $0$, and an integer value of \verb|I| $0$ when the given operand is any integer value different from \verb|I| $0$.

\verb|to_bool| is a function that yields an Isabelle predefined boolean given a value.
This function is used to compute short-circuit evaluation on the \verb|And| and \verb|Or| operations.
This function will fail whenever the parameter is anything different from an integer.
Upon successful execution the function will yield \verb|False| when the given parameter has a value equal to \verb|I| $0$, it will yield \verb|True| otherwise.

\verb|eq_val| is the function that computes the value of equality comparison between to values.
This function will fail whenever anything different than two integers or two addresses are given as parameters to the function.
Upon successful execution with two integer values the function will yield an integer value of \verb|I| $1$ if both operands were equal and an integer value of \verb|I| $0$ otherwise.
Upon successful execution with two address values the function will yield an integer value of \verb|I| $1$ if both addresses were equal and an integer value of \verb|I| $0$ otherwise.
Two addresses are regarded to as equal whenever both components of the address tuple are equal.

\verb|new_block| and \verb|load| are the functions explained earlier in section~\ref{subsection:syntax_expressions} that allocate a new block of memory and load a value of memory, respectively.


\section{Commands}\label{section:commands}

In this next section we will discuss in detail the syntax and semantics of Chloe's commands as well as functions and programs written in the language and we will discuss some restrictions.

\subsection{Syntax}\label{subsection:syntax_commands}

Chloe contains the following constructs: assignments, sequential composition, conditionals, while loops, SKIP which is equivalent to a noop because it does nothing (we use it to be able to express other syntactic constructs such as a contitional without an \verb|ELSE| branch), deallocation of memory, return statements and functions.
The expressions are the ones described in the previous section (~\ref{section:expressions}).

Here we proceed to describe the \textit{abstract syntax} for the commands of the Chloe language.

\begin{figure}
  \begin{lstlisting}[frame=single, mathescape=true]
  type_synonym fname = string

  datatype
    com = SKIP
        | Assignl lexp exp
        | Assign  vname exp
        | Seq     com  com
        | If      exp com com
        | While   exp com
        | Free    lexp
        | Return exp
        | Returnv
        | Callfunl lexp fname "exp list"
        | Callfun vname fname "exp list"
        | Callfunv fname "exp list"
  \end{lstlisting}

  \caption{Chloe commands}
  \label{fig:chloe_commands}
\end{figure}

In figure~\ref{fig:chloe_commands} we can find the definition of the datatype created in Isabelle for the commands, where \verb|lexp| and \verb|exp| stand for expressions described in the previous sections (~\ref{section:expressions}), \verb|vname| stands for variable names and \verb|fname| stands for the function names.

For assignment commands we define two different commands, one of them allows assignment to a variable, whereas the other one allows assignment to an address location in memory.
We need these two commands since our domain for addresses and integer values is disjoint, therefore an address cannot represent an integer value and viceversa.

We also have two return commands, one of them is for returning from functions with a return value, whereas the other one is for returning from a procedure which has a void return value.

Finally we have three different statements for function calling.
One of them (\verb|Callfunv|) is for functions without a return value.
The other two depend on the return location of the function, if the return location is a variable we use the \verb|Callfun| command and if the return location is a cell in memory we use the \verb|Callfunl| command.

In Isabelle we have defined a concrete infix syntax as well, which facilitates writing and reading commands in Chloe.
In table~\ref{tab:concrete_syntax} we introduce the concrete syntax supposing we have $x$ that ranges over variable names, $a$ that ranges over expressions, $c$, $c_{1}$ and $c_{2}$ that range over commands, $y$ that ranges over the LHS expressions and $f$ that ranges over function names.
We will continue to use the concrete syntax throughout the rest of this work to make it more readable.

\begin{table}[h!]
\centering
\begin{tabular}{|c|c|}
  \hline
  \textbf{Abstract syntax} & \textbf{Concrete syntax} \\ [0.5ex]
  \hline \hline
  \verb|Assignl| $y$ $a$ & $y$ \verb|::==| $a$ \\
  \verb|Assign| $x$ $a$ & $x$ \verb|::=| $a$ \\
  \verb|If| $a\ c_{1}\ c_{2}$ & \verb|IF| $a$ \verb|THEN| $c_{1}$ \verb|ELSE| $c_{2}$ \\
  \verb|While a| $c$ & \verb|WHILE| $a$ \verb|DO| $c$ \\
  \verb|Free| $y$ & \verb|FREE| $y$ \\
  \verb|Return| $a$ & \verb|RETURN| $a$ \\
  \verb|Returnv| & \verb|RETURNV| \\
  \verb|Calllfunl| $y\ f\ [a]$ & $y$ \verb|::==| $f$ \verb|(| $[a]$ \verb|)| \\
  \verb|Calllfun| $x\ f\ [a]$ & $x$ \verb|::=| $f$ \verb|(| $[a]$ \verb|)| \\
  \verb|Calllfunv| $f\ [a]$ & \verb|CALL| $f$ \verb|(| $[a]$ \verb|)| \\
  \hline
\end{tabular}

\caption{Abstract and concrete syntax equivalence}
\label{tab:concrete_syntax}
\end{table}


\section{Functions}\label{section:functions_commands}

In Chloe we have both functions that return values and those which dont have a return value.
For functions that do return a value we needed to figure out what to do with it.
We decided that the return value of every function should be assigned to a dummy variable or to a destination in a memory cell whenever a function was returning from a call.
We won't go into details explaining this design decision now but rather delay it until section~\ref{subsection:cfg} where we will be able to explain the reasoning behind this in a better way.

As we see in the definition in figure~\ref{fig:fun_def}, a function consists of a name, the formal parameters, the local variables and the body of the function, which is a, potentially big, command in the Chloe language.
We also define a predicate which checks whether a function declaration is valid or not.
A function declaration is considered valid if and only if the function parameters and the local variables have different names.

\begin{figure}
  \begin{lstlisting}[frame=single, mathescape=true]
  record fun_decl =
    name :: fname
    params :: vname list
    locals :: vname list
    body :: com

  valid_fun_decl :: fun_decl $\Rightarrow$ bool
  \end{lstlisting}

  \caption{Function definitions}
  \label{fig:fun_def}
\end{figure}

\section{Programs}\label{section:programs_commands}

A program in Chloe is conformed by a name, a list of global variables and a list of procedures as showed in figure~\ref{fig:prog_def}.
In that same figure we can see a definition that every valid program must comply with.

A program is considered valid if it complies with all the following conditions:

\begin{itemize}
  \item{The names for the global variables are different from one another.}
  \item{The names for the procedures in the program are different from one another.}
  \item{Every function declaration for every function in the program must be valid.}
  \item{The main procedure must be defined.}
  \item{None of the variable names or procedure names in the program must be a reserved keyword from C or a reserved keyword for testing.\footnote{Since we want to generate C code from the Chloe semantics we must guarantee that neither variable nor procedure names are any of the reserved C keywords or any of the reserved keywords used for testing variables.}}
  \item{The global variables and the procedure names in a program can't be the same.}
\end{itemize}



\begin{figure}
  \begin{lstlisting}[frame=single, mathescape=true]
  record program =
    name :: string
    globals :: vname list
    procs :: fun_decl list

  reserved_keywords =
    [''auto'', ''break'', ''case'', ''char'', ''const'', ''continue'',
     ''default'', ''do'', ''double'', ''else'', ''enum'', ''extern'',
     ''float'', ''for'', ''goto'', ''if'', ''inline'', ''int'', ''long'',
     ''register'', ''restrict'', ''return'', ''short'', ''signed'',
     ''sizeof'', ''static'', ''struct'', ''switch'', ''typedef'',
     ''union'', ''unsigned'', ''void'', ''volatile'', ''while'',
     ''_Bool'', ''_Complex'', ''_Imaginary'']

  test_keywords =
    [''num_tests'', ''passed'', ''failed'']

  definition valid_program :: program $\Rightarrow$ bool where
  valid_program p $\equiv$
      distinct (program.globals p)
    $\bigwedge$ distinct (map fun_decl.name (program.procs p))
    $\bigwedge$ ($\forall$ fd $\in$ set (progarm.procs p). valid_fun_decl fd)
    $\bigwedge$ ( let
         pt = proc_table_of p
       in
         ''main'' $\in$ dom pt
         $\bigwedge$ fun_decl.params (the (pt ''main'')) = [])
   $\bigwedge$ ( let
         prog_vars = set ((program.globals p) @
           collect_locals (program.procs p));
         proc_names = set (map (fun_decl.name) (program.procs p))
       in
         ($\forall$ name $\in$ prog_vars.
           name $\notin$ set (reserved_keywords @ test_keywords)) $\bigwedge$
         ($\forall$ name $\in$ proc_names.
           name $\notin$ set (reserved_keywords @ test_keywords)) $\bigwedge$
         ($\forall$ fname $\in$ proc_names.
           ($\forall$ vname $\in$ set (program.globals p). fname $\neq$ vname)))
  \end{lstlisting}

  \caption{Program definitions}
  \label{fig:prog_def}
\end{figure}


\section{Restrictions}\label{subsection:restrictions_commands}
Here we mention the architecture restrictions (64 bit architectures) this semantics assumes.
Not sure if int width should be explained here or before in the integer types.


\section{States}\label{section:states}

In simpler languages, which only support a limited set of features such as assignment, sequential composition, conditionals, loops and integer values, the state representation consists of simply a function that maps variables to values.
In Chloe that is not the case, by including constructs as functions, dynamic memory and pointers to our set of features, the state representation ceases to be a simple function that maps variable names to values.
In this section we will detail the complete components of the representation of a state.
Previously, in section~\ref{subsection:semantics_expressions} we began explaining a simplified ``visible state'', here we will clarify the difference between that visible state and a real state, as well as detailing the components of a state in a program.

\subsection{Valuation}\label{subsection:valuation}

As mentioned previously a valuation is simply a function that maps variable names to values.
Its return is of type \verb|val option option| which models a variable having one of the following states: undefined, uninitialized or holding a value.


\subsection{Stack}\label{subsection:stack}

Chloe supports functions calls, in order to do so an execution stack must be mantained in the state.
This execution stack (from now on we will refer to it as just the stack) is conformed by a list of stack frames.
Each of these stack frames contain important information about the current function call.

In figure~\ref{fig:stack_def} the type for a stack frame is a tuple containing a Chloe command, a valuation and a return location.
The Chloe command corresponds to the body of the function to be executed.
The valuation corresponds to the values of the variables that are local to the function that was called.
Finally the return location can be one of the following three: an address, a variable or an invalid return location.
Every function that returns a value has to return it to a dummy variable or to a cell in memory, those are what the variable and address return location indicate.
Finally we have an invalid return location for the functions that don't return a value.

The return location is set on the caller's stack frame, that is, when returning from a function the caller's stack frame is the one to be checked to know where to assign the return value or if a value is expected at all.
To further clarify this we can take this example code from figure~\ref{fig:stack_example}, it is a simple program where a function that adds the value of its two parameters is defined and then called from the main function.
Before the function call, the stack frame belonging to the main function has an $Invalid$ return location and after the function call, the return location changes to the variable $x$.
Notice that the stack frame that changes it's return location is the one belonging to main, this is because the caller is the one saving the return location where it expects to store the value of a returning function.
An $Invalid$ return value indicates the caller is not expecting to store any results, i.e.\ either the caller has not called any functions yet or the function it has called have no return value.

\begin{figure}
  \begin{lstlisting}[frame=single, mathescape=true]
  datatype return_loc = Ar addr | Vr vname | Invalid

  type_synonym stack_frame = com $\times$ valuation $\times$ return_loc
  \end{lstlisting}

  \caption{Stack definitions}
  \label{fig:stack_def}
\end{figure}


\begin{figure}
  \begin{subfigure}{0.3\textwidth}
  \begin{lstlisting}[language=c, mathescape=true]
  int sum(int a; int b){
    return a+b;
  }

  int main(){
    int x = 0;
    x = sum(2,2);
  }
  \end{lstlisting}
  \caption{Example code in C}
  \label{fig:stack_ex_c_code}
  \end{subfigure}
  \begin{subfigure}{0.3\textwidth}
    \begin{tabular}{|c|}
      \hline
      \hline
      $\vdots$ \\
      \\
      \hline
      $(x = \mathtt{sum}(2,2),$ \\
      $[x \mapsto 0],$ \\
      $ Invalid )$ \\
      \hline \hline
    \end{tabular}
  \caption{Stack before function call (Stack grows upwards)}
  \label{fig:stack_bef_fun_call}
  \end{subfigure}
  \begin{subfigure}{0.3\textwidth}
    \begin{tabular}{|c|}
      \hline
      \hline
      $\vdots$ \\
      \\
      \hline
      $(\mathtt{return}\ a+b,$ \\
      $[a \mapsto 2, b \mapsto 2],$ \\
      $ Invalid )$ \\
      \hline
      $(x = f(2,2),$ \\
      $[x \mapsto 0],$ \\
      $ x )$ \\
      \hline \hline
    \end{tabular}
  \caption{Stack after function call (Stack grows upwards)}
  \label{fig:stack_aft_fun_call}
  \end{subfigure}

  \caption{Calling convention code example}
  \label{fig:stack_example}
\end{figure}


\subsection{Procedure Table}\label{subsection:procedure_table}

Another extension we must add when dealing with functions is a procedure table.
A procedure table is defined as follows:

\begin{lstlisting}[mathescape=true, frame=single]
type_synonym proc_table = fname $\rightharpoonup$ fun_decl
\end{lstlisting}

where ``$\rightharpoonup$ \verb|fun_decl|'' is equivalent to writing ``$\Rightarrow$ \verb|fun_decl option|''.
This function maps function names to their declaration.

This function is constructed by taking the program definition and pairing every function declaration from the list of functions to its name.
Every program has its own procedure table.


\subsection{State}\label{subsection:state}

A state is defined as a tuple conformed by the stack, a valuation for the globals and the dynamic memory.

\begin{lstlisting}[mathescape=true, frame=single]
type_synonym state = stack_frame list $\times$ valuation $\times$ mem
\end{lstlisting}

\subsection{Initial State}\label{subsection:initial_state}

In order to build an initial state we need to define some components.
In figure~\ref{fig:prog_def} we find definitions for the initial configuration of the stack, the global variables and the memory.
The initial configuration for the stack consists of the stack frame for the main function of the program.
The initial configuration for the global valuation is a function where every possible variable name is mapped to the undefined variable value (\verb|None|).
The initial configuration for the dynamic memory is the empty memory, since nothing has been allocated.

Having defined all of these components the initial state configuration is given by the tuple consisting of the initial configuration for the stack, global variables and dynamic memory.


\begin{figure}
  \begin{lstlisting}[frame=single, mathescape=true]
  context fixes $program$ :: program begin

    private definition proc_table $\equiv$ proc_table_of program

    definition main_decl $\equiv$ the (proc_table ''main'')
    definition main_local_names $\equiv$ fun_decl.locals main_decl
    definition main_com $\equiv$ fun_decl.body main_decl

    definition initial_stack :: stack_frame list where
      initial_stack $\equiv$ [(main_com,
        map_of (map ($\lambda$. (x,None)) main_local_names),Invalid)]
    definition initial_glob :: valuation where
      initial_glob $\equiv$ map_of (map ($\lambda$. (x,None)) (program.globals program))
    definition initial_mem :: mem where initial_mem $\equiv$ []

    definition initial_state :: state where
      initial_state $\equiv$ (initial_stack, initial_glob, initial_mem)

  end
  \end{lstlisting}

  \caption{Initial state building}
  \label{fig:init_state_building}
\end{figure}



\subsection{Visible State}\label{subsection:visible_state}
We also define a visible state (as mentioned before in section~\ref{paragraph:visible_state}).
Executing a transformer function\footnote{we will cover transformer functions further in section~\ref{section:small_step}} over a state, excepting the transformer functions for function calls or returns, will be able to modify any part of the stack other than the current frame's local variables valuation.
This happens because we define the real transformations in the context of visible states and then lift the function to states.
Therefore, a transformer function over states, other than the ones for function call or return, cannot temper with the stack.

The stack frame on the top of the stack correspond to the current function that's being executed, and the command component of it is the Chloe command (or program) being executed.
Whenever a command is executed or a step is taken in our small-step semantics this command is updated to contain the command that is to be executed next.
In order to prove the small-step semantics is deterministic we must prove that whenever we have a non-empty execution stack the order in which we apply the evaluation transformer\footnote{This is the eval function lifted to work on states instead of visible states, we will discuss it in further detail in section~\ref{section:small_step}} and the function that updates the command is irrelevant to the resulting final state.
This is why we introduce a separate definition for a visible state apart from the regular state, it basically the same state with some limited information.

\section{Small Step Semantics}\label{section:small_step}

\begin{comment}
add a little summary here.
\end{comment}

\subsection{CFG}\label{subsection:cfg}

\paragraph{Enabled functions}\label{paragraph:enabled}

An enabled function is a partial function as follows:

\begin{lstlisting}[mathescape=true, frame=single]
type_synonym enabled = state $\rightharpoonup$ bool
\end{lstlisting}

it indicates whether a state is enabled to continue its execution.
This is a partial function, therefore its execution might fail.
The execution of an enabled function will fail whenever an error is encountered when evaluating the function and it yields a \verb|None| value indicating an erroneous state.

This function is useful for the execution of conditional constructs in our language.

Suppose we have the term ``\verb|IF| $b$ \verb|THEN| $c_{1}$ \verb|ELSE| $c_{2}$'' depending on the result of the enabled function we will decide whether the term reduces to $c_{1}$ or $c_{2}$.
The only cases where an execution wouldn't be enabled to continue is when a condition for an \verb|If| construct evaluates to \verb|True| and it would be attempting to take the \verb|False| branch or viceversa.
This is the only interesting case given the set of commands Chloe supports, for every other command the enabled function will always yield true.
In figure~\ref{fig:enabled_def} is the list of enabled functions we use.
The function \verb|truth_value_of| maps a value to a real boolean type, namely \verb|True| or \verb|False|.
We also find \verb|en_always| which always yields \verb|True|, the \verb|en_pos| function which only yields \verb|True| when the truth value of the expression given as a parameter evaluates to \verb|True| and the \verb|en_neg| function which only yields \verb|True| when the truth value of the expression given as a parameter evaluates to \verb|False|.
These functions will be used later in the CFG definition.

\begin{figure}
  \begin{lstlisting}[frame=single, mathescape=true]
  fun truth_value_of :: val $\Rightarrow$ bool where
    truth_value_of NullVal $\longleftrightarrow$ False
  | truth_value_of (I $i$) $\longleftrightarrow$ $i$ $\neq$ $0$
  | truth_value_of (A _) $\longleftrightarrow$ True

  abbreviation en_always :: enabled where en_always $\equiv\ \lambda$_. Some True

  definition en_pos :: exp $\Rightarrow$ enabled

  definition en_neg :: exp $\Rightarrow$ enabled
  \end{lstlisting}

  \caption{Enabled functions}
  \label{fig:enabled_def}
\end{figure}


\paragraph{Transformer functions}\label{paragraph:transformer}

A transformer function is a partial function as follows:

\begin{lstlisting}[mathescape=true, frame=single]
type_synonym transformer = state $\rightharpoonup$ state
\end{lstlisting}

it is a partial function that tranforms a state into another.
This is a partial function, therefore its execution might fail.
The execution of a transformer function will fail whenever an error is encountered at some point of its execution and yield a \verb|None| value indicating an erroneous state.

We define functions that will return a transformer function for each command in Chloe.
These functions will be used later in the CFG definition and are defined in figure~\ref{fig:transformer_def}
We define a transformer function \verb|tr_id| that will serve as the id function and just yields the same state it is given as a parameter.

The definitions listed in figure~\ref{fig:transformer_def} yield a transformer function which we will use when executing a command.
We proceed to roughly describe the effect the transformer functions yielded will have when applied to a state.

First of all we have the transformer for an assignment yielded by \verb|tr_assign|, it will evaluate the expression we want to assign and then proceed to perform a \verb|write| operation to the state and it yields the state resulting from this evaluation and \verb|write| operation.

The transformer for an assignment to a cell in the dynamic memory yielded by \verb|tr_assignl| will first evaluate the LHS expression to obtain the location in memory where the value will be updated, then it will evaluate the expression to obtain the new value and proceed to store the value in memory, it yields the state resulting from the evaluations and the \verb|store| operation.

The transformer for an evaluation yielded by \verb|tr_eval| will evaluate the expression and yield the state resulting from the evaluation.

The transformer for a free operation yielded by \verb|tr_free| will evaluate the LHS expression it receives as a parameter in order to obtain an address, and the block for this address will be deallocated, it yields the state resulting from the evaluation and deallocation of memory.

When performing a function call one must check the that formal parameters and the actual parameters given to the function have the same type and that each formal parameter has a corresponding given parameter.
Since we don't have a type system we will only check the second condition mentioned.
Also, we fix the evaluation order for the parameters given to a function as left to right order.
The parameters will always be evaluated following a left to right order.
Finally when calling a function we must also map the formal parameters to the values of the given parameters and regard these as local variables in the scope of our function.

We have a \verb|call_function| which yields a transformer function for any function call, this transformer checks that the number of formal parameters and given parameters is the same, it evaluates the given parameters from left to right, it creates a new stack frame containing the body of the function, the local variables valuation (which includes the parameters mapped to their given values and the local variables mapped to an uninitialized value) and the return location \verb|Invalid|, it yields the state resulting from performing these operations over the state.

When calling a function the caller has to change its stack frame in order to update the return location value in the current stack frame.
We define different functions that perform that change depending on the type of function call and then proceed to yield a transformer by calling the \verb|call_function| function.

\begin{comment}
I should also explain here why function calls always return to a dummy variable, right?
\end{comment}

These functions are \verb|tr_callfunl|, \verb|tr_callfun| and \verb|tr_callfunv|.
We use \verb|tr_callfunl| when the function's return location is a cell in memory.
In this case the function first evaluates the LHS expression to obtain the return addresss and updates the return location of the stack frame with it before calling \verb|call_function|, the resulting transformer function will yield a state resulting from this evaluation, the update of the stack frame and the operations done by \verb|call_function|.
It is important here that the evaluation of the LHS expression that yields an address value must be the first one done, this is in order to avoid unwanted behaviour since the function could change the state.

\begin{comment}
You can do a better job at clarifying this!
\end{comment}

We use \verb|tr_callfun| when the function's return location is a variable.
In this case the function updates the return location of the stack frame with the variable name before calling \verb|call_function|, the resulting transformer function will yield a state resulting from this update of the stack frame and the operations done by \verb|call_function|.

We use \verb|tr_callfunl| when the function is not expected to return a value.
In this case the function updates the return location of the stack frame with \verb|Invalid| before calling \verb|call_function|, the resulting transformer function will yield a state resulting from this update of the stack frame and the operations done by \verb|call_function|.

We have a \verb|tr_return| which yields a transformer function for a return call returning an expression in a function, this transformer will pop the last stack frame in the stack belonging to the returning functin, evaluate the value corresponding to the expression returned by the function and if the stack is not empty then proceed to retrieve the return location and depending on if it is an address, a variable or an invalid location it will yield the state resulting from storing the value in memory, writing the value to a variable or yield the state as it is, respectively as well as popping the last stack frame of the stack.
Note that if the function returns a value but its return location is expected to be invalid then the returned value is ignored instead of being considered an erroneous execution.

Finally we have a \verb|tr_return_void| which yields a transformer function for a return call in a function, this transformer will pop the last stack frame belonging to the returning function and if the stack is not empty then proceed to get the return location and if it is anything different from an invalid return location proceed to return a \verb|None| value which represents an error.
If, on the other hand, the return location is in fact an invalid one then it will yield the state resulting from the last frame of the stack being popped.

\begin{comment}
FIXME
\end{comment}

\begin{figure}
  \begin{lstlisting}[frame=single, mathescape=true]
  abbreviation (input) tr_id :: transformer where tr_id $\equiv$ Some

  tr_assign :: vname $\Rightarrow$ exp $\Rightarrow$ transformer
  tr_assignl :: lexp $\Rightarrow$ exp $\Rightarrow$ transformer
  tr_eval :: exp $\Rightarrow$ transformer
  tr_free :: lexp $\Rightarrow$ transformer
  call_function :: proc_table $\Rightarrow$ fname $\Rightarrow$ exp list $\Rightarrow$ transformer
  tr_callfunl :: proc_table $\Rightarrow$ lexp $\Rightarrow$ fname $\Rightarrow$ exp list $\Rightarrow$ transformer
  tr_callfun :: proc_table $\Rightarrow$ vname $\Rightarrow$ fname $\Rightarrow$ exp list $\Rightarrow$ transformer
  tr_callfunv :: proc_table $\Rightarrow$ fname $\Rightarrow$ exp list $\Rightarrow$ transformer
  tr_return :: exp $\Rightarrow$ transformer
  tr_return_void :: transformer
  \end{lstlisting}

  \caption{Transformer functions}
  \label{fig:transformer_def}
\end{figure}


\paragraph{CFG}
A Control Flow Graph is a graph representation that covers the different paths a program can take during its execution.
In our CFG the nodes represent commands and the edges are labeled with a tuple containing an enabled function and a transformer function.
The meaning is that execution moves a command from node to node upon successful application of the enabled and transformer functions on the edge label over the state.


\begin{figure}
  \begin{lstlisting}[frame=single, mathescape=true]
  type_synonym cfg_label = enabled $\times$ transformer

  inductive cfg :: com $\Rightarrow$ cfg_label $\Rightarrow$ com $\Rightarrow$ bool where

    Assign: cfg ($x$ ::= $a$) (en_always,tr_assign $x$ $a$) SKIP
  | Assignl: cfg ($x$ ::==  $a$) (en_always,tr_assignl $x$ $a$) SKIP
  | Seq1: cfg (SKIP;; $c_{2}$) (en_always, tr_id) $c_{2}$
  | Seq2: $[\![$cfg $c_{1}$ $a$ $c_{1}'$ $]\!]$ $\Longrightarrow$ cfg ($c_{1}$;; $c_{2}$) $a$ ($c_{1}'$;; $c_{2}$)
  | IfTrue: cfg (IF $b$ THEN $c_{1}$ ELSE $c_{2}$) (en_pos $b$, tr_eval $b$) $c_{1}$
  | IfFalse: cfg (IF $b$ THEN $c_{1}$ ELSE $c_{2}$) (en_neg $b$, tr_eval $b$) $c_{2}$
  | While: cfg (WHILE $b$ DO $c$) (en_always, tr_id)
      (IF $b$ THEN $c$;; WHILE $b$ DO $c$ ELSE SKIP)
  | Free: cfg (FREE $x$) (en_always, tr_free $x$) SKIP

  | Return: cfg (Return $a$) (en_always, tr_return $a$) SKIP
  | Returnv: cfg Returnv (en_always, tr_return_void) SKIP

  | Callfunl: cfg (Callfunl $e$ $f$ params)
      (en_always, tr_callfunl proc_table $e$ $f$ params) SKIP
  | Callfun: cfg (Callfun $x$ $f$ params)
      (en_always, tr_callfun proc_table $x$ $f$ params) SKIP
  | Callfunv: cfg (Callfunv $f$ params)
      (en_always, tr_callfunv proc_table $f$ params) SKIP
  \end{lstlisting}

  \caption{CFG rules}
  \label{fig:cfg_rules}
\end{figure}

We can see an inductive definition in figure~\ref{fig:cfg_rules} for CFG, where we can see the rules to form edges between commands.
An assignment to a variable can be reduced to the \verb|SKIP| command by an edge that is always enabled and the transformer function to update the state is given by the result of \verb|tr_assign| called with the parameters specific for the command.

An assignment to a cell in memory can also be reduced to \verb|SKIP| by an edge that is always enabled and the transformer function to update the state is given by the result of \verb|tr_assignl| called with the parameter specific for the command.

The sequential composition has two different ways in which it can be reduced.
When the first command in the sequential composition is a \verb|SKIP| command we can reduce the whole command to just the second command by an edge that is always enabled and that the transformer over the state is the \verb|tr_id|.
For the second case of sequential composition \verb|(| $c_{1}$ \verb|;;| $c_{2}$ \verb|)| where $c_{1}$ is not \verb|SKIP|, if we have a reduction from $c_{1}$ to some other command $c_{1}'$ with some edge label $a$ containing an enabled and a transformer function, then we can reduce from \verb|(| $c_{1}$ \verb|;;| $c_{2}$ \verb|)| to \verb|(| $c_{1}'$ \verb|;;| $c_{2}$ \verb|)| with the same $a$ edge label.

We also have two cases when it comes to the case of the conditional.
We will have a reduction for taking true branch and another one for taking the false branch.
In the case where we take the true branch we can reduce \verb|IF| $b$ \verb|THEN| $c_{1}$ \verb|ELSE| $c_{2}$ to $c_{1}$, by an edge with an enabled function given by \verb|en_pos| which will be enabled only when the condition $b$ evaluates to \verb|True| and with a transformer function given by \verb|tr_eval|.
In the case where we take the false branch we can reduce \verb|IF| $b$ \verb|THEN| $c_{1}$ \verb|ELSE| $c_{2}$ to $c_{2}$, by an edge with an enabled function given by \verb|en_neg| which will be enabled only when the condition $b$ evaluates to \verb|False| and with a transformer function given by \verb|tr_eval|.

In the case of a loop it can be always reduced from \verb|WHILE| $b$ \verb|DO| $c$ to \verb|IF| $b$ \verb|THEN| $c$ \verb|;; WHILE| $b$ \verb|DO| $c$ \verb|ELSE SKIP| by unrolling the loop once, the edge will have an enabled function that's always enabled and the transformer over the state is the \verb|tr_id|.

The \verb|FREE| $x$ command can be reduced to \verb|SKIP|, by an edge which is always enabled and a transformer function given by the result of \verb|tr_free| called with $x$.

Both return commands can be reduced to \verb|SKIP|, by an edge which is always enabled and a transformer function given by the result of \verb|tr_return| or \verb|tr_return_void| depending on which return command it is.

Finally all function calls can be reduced to \verb|SKIP|, by an edge which is always enabled and a transformer function given by the result of either \verb|tr_callfunl|, \verb|tr_callfun| or \verb|tr_callfunv| depending on whether it is a function call that returns to a memory cell, a variable or returns no value.

\subsection{Small Step semantics rules}\label{subsection:rules_small_step}

Finally, we introduce the rules for the small-step semantics for Chloe.
A small-step semantics is chosen over a big-step definition for the semantics due to the fact that we want a finer grained semantics.
The big-step semantics has a major drawback, which is that it cannot differentiate between a non-terminating execution and getting stuck in an erroneous configuration.
This is why we prefer a more detailed semantics that allows us to differentiate between nontermination and getting stuck because it allows us to talk about intermediate states during evaluation.

Usually a configuration in a small step semantics is a pair conformed by a command and a state.
Since we are working with procedures and we have the command that is being executed in the stack frame our small-step definition is defined over states.
A small, atomic step can be taken from one state to another.

The small-step rules for the semantics are detailed in figure~\ref{fig:small_step_rules}.
The infix syntax for the small step semantics is written as $s \rightarrow s_{2}$ which means we take a small step from $s$ to $s_{2}$.
A step can be taken if the following conditions are met:

\begin{itemize}
  \item{The stack is not empty.}
  \item{There is a CFG edge between $c_{1}$ and $c_{2}$.}
  \item{The command at the topmost stack frame in the initial state is $c_{1}$.}
  \item{Applying the enabled function over the state yields \verb|True|.}
  \item{Applying the transformer function to the state with the command at the topmost stack frame updated from $c_{1}$ to $c_{2}$ yields a new state $s_{2}$.}
\end{itemize}

Given that all the previous conditions are fulfilled then a small step can go from state $s$ to $s_{2}$.

A small step can also be taken upon return from a function which returns no value.
If the command at the top of the stack is \verb|SKIP|, the stack is not empty and applying the transformer function on the initial state $s$ yields a new state $s_{2}$, then we can take a small step from $s$ to $s_{2}$

The type for \verb|small_step| is from \verb|state| to \verb|state option|, the second configuration is enclosed in an option type because taking a small step can result in an erroneous state in which we will get stuck.

A small step can fail to be taken in any of the following cases:

\begin{itemize}
  \item{Either the enabled function or the transformer function yield \verb|None| when evaluated over the initial state.
  This indicates an erroneous state was reached when evaluating one of those functions.
  We propagate the erroneous state by taking a small step from state $s$ to \verb|None| and then the execution will get stuck there.}
  \item{If applying the transformer function \verb|tr_return_void| over the state yields a \verb|None| value, the command at the top of the stack in the initial state is \verb|SKIP| and the stack is not empty, then we also propagate this erroneous \verb|None| state by taking a small step from $s$ to None.
  This indicates there was an error returning from a function without a return value.}
\end{itemize}

\begin{figure}
  \begin{lstlisting}[frame=single, mathescape=true]
  inductive
    small_step :: state $\Rightarrow$ state option $\Rightarrow$ bool (infix $\rightarrow$ 55)
  where
    Base: $[\![$ $\neg$ is_empty_stack $s$; $c_{1}$=com_of $s$; cfg $c_{1} (en, tr) $c_{2};
      en $s$ = Some True; tr (upd_com $c_{2}$ $s$) = Some $s_{2}$ $]\!]$ $\Longrightarrow$ s $\rightarrow$ Some $s_{2}$
  | None: $[\![$ $\neg$ is_empty_stack $s$; $c_{1}$=com_of $s$; cfg $c_{1} (en, tr) $c_{2};
      en $s$ = None $\vee$ tr (upd_com $c_{2}$ $s$) = None$]\!]$ $\Longrightarrow$ s $\rightarrow$ None
  | Return_void: $[\![$ $\neg$ is_empty_stack $s$; com_of $s$ = SKIP;
      tr_return_void $s$ = Some $s'$ $]\!]$ $\Longrightarrow$ $s$ $\rightarrow$ Some $s'$
  | Return_void_None: $[\![$ $\neg$ is_empty_stack $s$; com_of $s$ = SKIP;
      tr_return_void $s$ = None $]\!]$ $\Longrightarrow$ $s$ $\rightarrow$ None

  inductive
    small_step' :: (state) option $\Rightarrow$ (state) option $\Rightarrow$ bool (infix $\rightarrow'$  55)
  where
    $s\ \rightarrow\ s'\ \Longrightarrow$ Some $s\ \rightarrow'\ s'$

  abbreviation
    small_steps :: (state) option $\Rightarrow$ (state) option $\Rightarrow$ bool (infix $\rightarrow*$ 55)
  where $s_{0}$ $\rightarrow*$ $s_{f}$ ==  star small_step' $s_{0}$ $s_{f}$
  \end{lstlisting}

  \caption{Small-step rules}
  \label{fig:small_step_rules}
\end{figure}

We just defined how to take a single step in our semantics.
In order to take more than one step and define the execution of a program in our semantics we must lift the definition of \verb|small_step| to \verb|state option| in both the initial and the final state.

We also define (in figure~\ref{fig:small_step_rules}) a new \verb|small_step'| based on our previous definition of \verb|small_step|, this definition essentially says that if a small step can be taken from state $s$ to state $s'$ then a small step can be taken from \verb|Some| $s$ to $s'$.
Note that $s'$ can either be a \verb|None| value or a \verb|Some| $s_{i}$.
The infix syntax for this new \verb|small_step'| is written as \verb|Some| $s \rightarrow' s'$, which means we take a small step from \verb|Some| $s$ to $s'$.

This way we have lifted our small step definition to the type \verb|state option| and we can define the execution of a program as the reflexive transitive closure of the newly defined \verb|small_step'|, using Isabelle's \verb|star| operator.
The infix syntax for this is written as $s_{0} \rightarrow* s_{f}$, which means we can go from state $s_{0}$ to $s_{f}$ in zero or more small steps.


\subsubsection{Determinism}

We want to prove our semantics to be deterministic.
In order to go through with that proof a set of lemmas must be first defined and proved.
We will proceed to explain such defined lemmas for our semantics in this section.

This first lemma indicates that whenever we have a command different from \verb|SKIP| there is either always an enabled action to take in the CFG or an error ocurrs when trying to evaluate the enabled function.

\begin{lemma}[CFG has enabled action]
$\newline$
$c$ $\neq$ \verb|SKIP| $\Longrightarrow$ $\exists\ c'\ en\ tr$. \verb|cfg| $c\ (en,tr)\ c'\ \wedge\ (en\ s\ =$ \verb|None| $\vee\ en\ s\ =$ \verb|Some True|$)$
\label{lemma:cfg_enabled_action}
\end{lemma}

\begin{proof}
The proof is by induction on the command.
Except for two cases the proof is solved automatically.
Those interesing cases are the \verb|Seq| and the \verb|If| cases.
In the \verb|Seq| case we need to make a further case distinction on the first command of the sequential composition in order to differentiate the cases where it is \verb|SKIP| and when it is not.
In both cases there is a CFG rule that ensures there is an enabled action and the case is solved automatically.

In the \verb|If| case we need to make a case distinction over the value of $\mathtt{en\_pos}\ b\ s$.
It can fail and return a None value, in which case the case is solved.
If it doesn't fail then we must check the boolean value returned by the \verb|en_pos| function.
In the case where it is \verb|True| then the case is solved.
The tricky case comes when the evaluation of \verb|en_pos| is \verb|False|, semantically, this means the else branch should be taken instead.
Therefore when $\mathtt{en\_pos}\ b\ s$ is \verb|False|, $\mathtt{en\_neg}\ b\ s$ will always evaluate to \verb|True|, this means that the command will have an enabled action, namely $\mathtt{en\_neg}\ b\ s$ and the proof of this case is complete.
\end{proof}

Next we want to prove that as long as the stack is not empty the small-step semantics can always take a step.
We will make use of the previous lemma in the proof for this new lemma.

\begin{lemma}[Semantics can take a step]
$\newline$
$\neg$ \verb|is_empty_stack| $s$ $\Longrightarrow$ $\exists\ x.\ s\ \rightarrow\ x$
\label{lemma:can_take_step}
\end{lemma}

\begin{proof}
From the assumptions we know that the stack is not empty, therefore we can rewrite the state as follows: $s = ((c, locals, rloc)\#\sigma,\gamma,\mu)$.
We can then do a proof by cases.
We have the case where $c\ =\ \mathtt{SKIP}$ and the case where $c\\neq\ \mathtt{SKIP}$.

The case where $c\ =\ \mathtt{SKIP}$ is the case where we are returning from the execution of a function.
We know that the stack is not empty, that $c\ =\ \mathtt{SKIP}$ and that $s = ((c, locals, rloc)\#\sigma,\gamma,\mu)$.
This case corresponds to the \verb|Return_void| and \verb|Return_void_None| rules in the definition of \verb|small_step|.
In both these cases the semantics can take a step, either to some new state $s'$ or to \verb|None|.
This goal is solved automatically by Isabelle indicating the mentioned rules and assumptions.

The second case is where $c\\neq\ \mathtt{SKIP}$.
This is the case when we are executing any other command and not returning from a function.
In this case we use the previous lemma~\ref{lemma:cfg_enabled_action} and from that we know that $\mathtt{cfg}\ c\ (en,tr)\ c'$ and either the enabled function fails ($\mathtt{en}\ s\ = \mathtt{None}$) or it is enabled ($\mathtt{en}|\ s\ = \mathtt{Some True}$).

We proof this subgoal by splitting it into cases.
In the first case we assume $\mathtt{cfg}\ c\ (en,tr)\ c'$ and $\mathtt{en}\ s\ = \mathtt{None}$.
This is the case where the evaluation function fails, we know this case takes a step to None because of the \verb|None| case of the small step definition.
Then we have proved that it exists a state so that $s\ \rightarrow\ \mathtt{None}$.

In the next case we assume $\mathtt{cfg}\ c\ (en,tr)\ c'$ and $\mathtt{en}\ s\ = \mathtt{Some True}$.
This is the case where we are enabled to take a step, we must still check two more cases.
Applying the transformer function over the state updated with the command ($tr\ (upd\_com\ c'\ s)$) can either fail or not.
In the case it fails (it returns \verb|None|) we can take a step to \verb|None| and we will have proved that there is a state to which $s$ can take a small step, namely \verb|None|.
In the case it doesn't fail, it will return $\mathtt{Some}\ s_{2}$, in which case we can take a step to $\mathtt{Some}\ s_{2}$ and we will have proved that there is a state to which $s$ can take a small step, namely $\mathtt{Some}\ s_{2}$.
\end{proof}

We define several lemmas that will come in handy later on.

\begin{lemma}[CFG gets stuck at SKIP]
$\newline$
$\neg$ \verb|cfg SKIP| $a\ c$
\label{lemma:stuck_at_skip}
\end{lemma}

\begin{proof}
The property is proved automatically.
\end{proof}

\begin{lemma}[Small step gets stuck on an empty stack]
$\newline$
\verb|is_empty_stack| $s$ $\Longrightarrow$ $\neg$ $(s\ \rightarrow\ cs')$
\label{lemma:ss_empty_stack_stuck}
\end{lemma}

\begin{proof}
The property is proved automatically.
\end{proof}

\begin{lemma}[Small step' gets stuck on an empty stack]
$\newline$
\verb|is_empty_stack| $s$ $\Longrightarrow$ $\neg$ $(Some s\ \rightarrow\ cs')$
\label{lemma:ss'_empty_stack_stuck}
\end{lemma}

\begin{proof}
The property is proved automatically.
\end{proof}

Lemmas~\ref{lemma:ss_empty_stack_stuck} and~\ref{lemma:ss'_empty_stack_stuck} define the final state in which the semantics will get stuck, the semantics will get stuck when there are no more stack frames in the stack.

\begin{lemma}[en\_neg has the opposite result as en\_pos]
$\newline$
\verb|en_neg| $e\ s$ = \verb|map_option (HOL.Not) (en_pos| $e\ s$ \verb|)|
\label{lemma:en_neg_by_pos}
\end{lemma}

\begin{proof}
The property is proved automatically unfolding the definitions of \verb|en_neg| and \verb|en_pos|.
\end{proof}

Lemma~\ref{lemma:en_neg_by_pos} states that everytime the enabled function \verb|en_neg| has a value (different than None) over a state, \verb|en_pos| will have the same but opposite result.
In the case where one of the functions fails the other one will too, and they will yield \verb|None|.
If they don't fail their results will be opposite, that is, if one has \verb|Some True| as a result, the other one will have \verb|Some False| as a result.
This lemma will come in handy when proving determinism.

\begin{lemma}[CFG is deterministic]
$\newline$
$\mathtt{cfg}$ $c$ $a1$ $c'$ $\wedge$
$\mathtt{cfg}$ $c$ $a2$ $c''$
$\newline$
$\Longrightarrow$
$a1\ =\ a2$ $\wedge$ $c'\ =\ c''$ $\vee$
$\newline$
$\exists\ b.\ a1\ = (\mathtt{en\_pos}\ b,\ \mathtt{tr\_eval}\ b)\ \wedge\ a2\ = (\mathtt{en\_neg}\ b,\ \mathtt{tr\_eval}\ b)\ \vee$
$\newline$
$\exists\ b.\ a1\ = (\mathtt{en\_neg}\ b,\ \mathtt{tr\_eval}\ b)\ \wedge\ a2\ = (\mathtt{en\_pos}\ b,\ \mathtt{tr\_eval}\ b)$
\label{lemma:cfg_determ}
\end{lemma}

\begin{proof}
The proof is by induction on the command, with the cases of the \verb|cfg| rules generated by Isabelle all cases are solved automatically.
\end{proof}

Lemma~\ref{lemma:cfg_determ} states that CFG is deterministic.
The only case where it is not is in the coditional case, for which we add an extra alternative in the conclusion.
It can happen that we have a CFG rule starting at an \verb|If| command that has an edge to a $c_{1}$ and also an edge to $c_{2}$.
This is not really a problem since the enabled functions guarantee that whenever that happens only one of the branches can be taken.


\begin{lemma}[Update command and transformer function independence]
$\newline$
$\neg\ \mathtt{is\_empty\_stack}\ s\ \Longrightarrow$
$\newline$
$\mathtt{lift\_transformer\_nr}\ tr\ (\mathtt{upd\_com}\ c\ s)\ =$
$\newline$
$\mathtt{map\_option}\ (\mathtt{upd\_com}\ c)\ (\mathtt{lift\_transformer\_nr}\ tr\ s)$
\label{lemma:lift_upd_com}
\end{lemma}

\begin{proof}
It is proved automatically by unfolding the definition of \verb|lift_transformer_nr|.
\end{proof}

\begin{lemma}[Update command and evaluation transformer function independence]
$\newline$
$\neg\ \mathtt{is\_empty\_stack}\ s\ \Longrightarrow$
$\newline$
$\mathtt{tr\_eval}\ e\ (\mathtt{upd\_com}\ c\ s)\ =$
$\newline$
$\mathtt{map\_option}\ (\mathtt{upd\_com}\ c)\ (\mathtt{tr\_eval}\ e\ s)$
\label{lemma:tr_eval_upd_com}
\end{lemma}

\begin{proof}
It is proved automatically by unfolding the definition of \verb|tr_eval|.
\end{proof}

The \verb|lift_transformer_nr| function lifts to states the definition of a transformer function that operates in the visible state level.
Lemmas~\ref{lemma:lift_upd_com} states that it does not matter in which order a transformer function is applied over a state and the command in the top of the stack frame is updated, since it will always yield the same result.
This is because the update command function only modifies the command at the top of the stack and the transformer function cannot access and modify that part of the stack since it only modifies the visible state.

Lemma~\ref{lemma:tr_eval_upd_com} is a more specific version of the previous lemma that states the same fact specifically for the evaluation transformer function.


All the previous lemmas and definitions have been building up to this next proof, the determinism of the small step semantics.
\begin{lemma}[Small step is deterministic]
$\newline$
$s\ \rightarrow\ s'\ \wedge\ s\ \rightarrow\ s''\ \Longrightarrow\ s'\ =\ s''$
\label{lemma:small_step_determ}
\end{lemma}

\begin{proof}
The proof is by cases on the small step semantics.
We obtain 4 cases, each corresponding to each rule in the small step semantics.
The goals generated by the \verb|Return_void| and \verb|Return_void_None| rules are solved automatically.
The goals generated by the \verb|Base| and the \verb|None| are solved automatically after adding lemmas~\ref{lemma:en_neg_by_pos} and~\ref{lemma:tr_eval_upd_com}.
\end{proof}

Then we are only left to show that \verb|small_step'| is also deterministic.

\begin{lemma}
$\newline$
$s\ \rightarrow'\ s'\ \wedge\ s\ \rightarrow'\ s''\ \Longrightarrow\ s'\ =\ s''$
\label{lemma:small_step'_determ}
\end{lemma}

\begin{proof}
The proof is by cases on the small step semantics.
It is proved automatically by using lemma~\ref{lemma:small_step_determ}.
\end{proof}


\section{Interpreter}\label{section:interpreter}

\subsection{Single step}\label{subsection:single_step}

\begin{figure}
  \begin{lstlisting}[frame=single, mathescape=true]
  datatype cfg_edge = Base $transformer$ $com$
                    | Cond $enabled$ $transformer$ $com$ $com$

  context fixes proc_table :: proc_table begin

    fun cfg_step :: "com $\Rightarrow$ cfg_edge" where
      "cfg_step SKIP = undefined"
    | "cfg_step ($x$ ::= $a$) = Base (tr_assign $x$ $a$) SKIP"
    | "cfg_step ($x$ ::== $a$) = Base (tr_assignl $x$ $a$) SKIP"
    | "cfg_step (SKIP;; $c_{2}$) = Base tr_id $c_{2}$"
    | "cfg_step ($c_{1}$;;$c_{2}$) = (case cfg_step $c_{1}$ of
        Base $tr$ $c$ $\Rightarrow$ Base tr (c;;c2)
      | Cond $en\ tr\ ca\ cb$ $\Rightarrow$ Cond $en\ tr\ \mathtt{(}ca\mathtt{;;}c2\mathtt{)}\ \mathtt{(}cb\mathtt{;;}c2\mathtt{)}$
       )"
    | "cfg_step (IF $b$ THEN $c_{1}$ ELSE $c_{2}$) = Cond (en_pos $b$) (tr_eval $b$) $c_{1}$ $c_{2}$"
    | "cfg_step (WHILE $b$ DO $c$) =
        Base tr_id (IF $b$ THEN $c$;; WHILE $b$ DO $c$ ELSE SKIP)"
    | "cfg_step (FREE $x$) = Base (tr_free $x$) SKIP"
    | "cfg_step (Return $a$) = Base (tr_return $a$) SKIP"
    | "cfg_step Returnv = Base (tr_return_void) SKIP"
    | "cfg_step (Callfunl $e$ $f$ params) =
        Base (tr_callfunl proc_table $e$ $f$ params) SKIP"
    | "cfg_step (Callfun $x$ $f$ params) =
        Base (tr_callfun proc_table $x$ $f$ params) SKIP"
    | "cfg_step (Callfunv $f$ params) =
        Base (tr_callfunv proc_table $f$ params) SKIP"

  end
  \end{lstlisting}

  \caption{Single step edges}
  \label{fig:single_step_edges}
\end{figure}

\begin{figure}
  \begin{lstlisting}[frame=single, mathescape=true]
  definition fstep :: proc_table $\Rightarrow$ state $\Rightarrow$ state option where
    fstep proc_table $s$ $\equiv$
      if com_of $s$ = SKIP then
        tr_return_void $s$
      else
        case cfg_step proc_table (com_of $s$) of
            Base $tr\ c'$ $\Rightarrow$ $tr$ (upd_com $c'\ s$)
          | Cond $en\ tr\ c1\ c2$ $\Rightarrow$ do {
              $b$ $\leftarrow$ en $s$;
              if $b$ then
                tr (upd_com $c1\ s$)
              else  
                tr (upd_com $c2\ s$)
            }
  \end{lstlisting}

  \caption{Definition of fstep}
  \label{fig:fstep_def}
\end{figure}

There are two kinds of edges in the CFG.
We create a new datatype for representing them.


A $Base$ edge that has a transformer function and is always enabled and takes us to a new command and a $Cond$ edge that apart from the transformer function also has an enabled function and two commands, the enabled function indicates whether we take a step to the first or the second command.
Also we define a \verb|cfg_step| function that given the starting command returns which kind of edge follows in the CFG.

The function \verb|fstep|, defined in figure~\ref{fig:fstep_def}, is how we take a single step in the executionof the semantics.
An execution of a step in the semantics will take us from an initial state to a new state which can be an erroneous state (\verb|None|) or a valid one (\verb|Some| $s$).
To execute a \verb|SKIP| command we call the transformer function for returning without a value from a function \verb|tr_return_void|.
Otherwise we will check which kind of step we should take by means of the \verb|cfg_Step| function and, based on that, decide what to do.
If it is a \verb|Base| step we will call the transformer function over the state with the updated command.
If, on the other hand, it is a \verb|Cond| step we will evaluate the condition and call the transformer function over the state with the updated command.

\subsubsection{Equality between small-step semantics and single-step execution}\label{subsubsection:equality_ss_ss}

We must now prove that the single step execution is semantically equivalent to taking a step in the small-step semantics.
This is proving that $\neg\ \mathtt{is\_empty\_stack}\ \Longrightarrow\ s\ \rightarrow\ s'\ \longleftrightarrow\ \mathtt{fstep}\ s\ =\ s'$\footnote{fstep has an extra parameter, namely the procedure table, which we will avoid writing here to make it simpler to read}.
We prove both directions of the equivalence separately: for every step taken in the small-step semantics, there is an equivalent step that can be taken in the execution by \verb|fstep| which will lead to the same final state and viceversa.


We start by showing that any step taken in the small-step semantics can be simulated by a step taken with fstep.

\begin{lemma}[($\Rightarrow$) direction of the proof.]
$\newline$
$s\ \rightarrow\ s'\ \Longrightarrow\ \mathtt{fstep}\ s\ =\ s'$
\label{lemma:fstep1}
\end{lemma}

\begin{proof}
This proof is done by induction over the small-step semantics.
\end{proof}

Then we consider the other direction:

\begin{lemma}[($\Leftarrow$) direction of the proof.]
$\newline$
$\neg\ \mathtt{is\_empty\_stack}\ s\ \Longrightarrow\ s\ \rightarrow\ (\mathtt{fstep}\ s)$
\label{lemma:fstep2}
\end{lemma}

\begin{proof}
\end{proof}

Both directions together (Lemma~\ref{lemma:fstep1} and lemma~\ref{lemma:fstep2}) let us then show the equivalence we were aiming for originally:

\begin{lemma}[Equivalence between a small step and a single step]
$\newline$
$\neg\ \mathtt{is\_empty\_stack}\ \Longrightarrow\ s\ \rightarrow\ s'\ \longleftrightarrow\ \mathtt{fstep}\ s\ =\ s'$
\label{lemma:equivalence_ss_ss'}
\end{lemma}


\subsection{Execution and Interpretation}\label{subsection:exec_interp}

In order to execute a program we will define an interpreter for it.

\begin{figure}
  \begin{lstlisting}[frame=single, mathescape=true]
  fun is_term :: "state option $\Rightarrow$ bool" where
    "is_term (Some $s$) = is_empty_stack $s$"
  | "is_term None = True"

  definition interp :: "proc_table $\Rightarrow$ state $\Rightarrow$ state option" where
    "interp proc_table $cs$ $\equiv$ (while
      (HOL.Not $\circ$ is_term)
      ($\lambda$Some $cs$ $\Rightarrow$ fstep proc_table $cs$)
      (Some $cs$))"
  \end{lstlisting}

  \caption{Definition of an interpreter for Chloe}
  \label{fig:interpreter_def}
\end{figure}

In figure~\ref{fig:interpreter_def} we find such definition.
First we define the criteria on which a state is considered final.
A state will be considered final when its execution stack is empty or when it is \verb|None|.

The interpreter for our semantics works as follows: As long as a final state is not reached we execute \verb|fstep|.

Finally we also show a lemma that states that if a state is final, then it is the result of the interpretation.

\begin{lemma}
$\newline$
$\mathtt{is\_term}\ \mathtt{(Some}\ cs\mathtt{)}\ \Longrightarrow\ \mathtt{interp}\ \mathtt{proc\_table}\ cs\ =\ \mathtt{Some}\ cs$
\label{lemma:interp_term}
\end{lemma}

In order to show this we need a lemma that unfolds the loop in the definition of our interpreter:

\begin{lemma}
$\newline$
$\mathtt{interp}\ \mathtt{proc\_table}\ cs = ($
$\newline$
$\mathtt{if}\ \mathtt{is\_term}\ (\mathtt{Some}\ cs)\ \mathtt{then}\ \mathtt{Some}\ cs$
$\newline$
$\mathtt{else}\ \mathtt{do\{}\ cs \leftarrow\ \mathtt{fstep}\ \mathtt{proc\_table}\ cs\mathtt{;}\ \mathtt{interp}\ \mathtt{proc\_table}\ cs$
$\mathtt{\}})$
\label{lemma:interp_unfold}
\end{lemma}

\begin{proof}
The proof is solved automatically.
\end{proof}

With lemma~\ref{lemma:interp_unfold} the proof for lemma~\ref{lemma:interp_term} is solved automatically.

Finally, only valid programs can be executed. In figure~\ref{fig:execute_def} we can see the definition for the function that executes a program.
In order to execute a program we assert that it is valid with the previously defined \verb|valid_program| and then proceed to interpret the initial state of the program p.

\begin{figure}
  \begin{lstlisting}[frame=single, mathescape=true]
  definition execute :: "program $\Rightarrow$ state option" where
    "execute $p$ $\equiv$ do {
      assert (valid_program $p$);
      interp (proc_table_of $p$) (initial_state $p$)
    }"
  \end{lstlisting}

  \caption{Definition of an interpreter for Chloe}
  \label{fig:execute_def}
\end{figure}


\subsection{Correctness}\label{subsection:correctness}

\begin{figure}
  \begin{lstlisting}[frame=single, mathescape=true]
  definition "yields $\equiv$ $lambda\ cs\ cs'$. Some $cs$ $\rightarrow*$ $cs'$ $/wedge$ is_term cs'"

  definition "terminates $\equiv$ $lambda\ cs$. $\exists\ cs'$. yields $cs\ cs'$"
  \end{lstlisting}

  \caption{Definitions about program execution}
  \label{fig:execution_definitions}
\end{figure}

\begin{lemma}
$\newline$
$\mathtt{None}\ \rightarrow*\ z\ \longleftrightarrow\ z\ =\ \mathtt{None}$
\label{lemma:none_star_preserved}
\end{lemma}

\begin{theorem}
$\newline$
$\mathtt{terminates}\ cs\ \Longrightarrow\ (\mathtt{yields}\ cs\ cs')\ \longleftrightarrow\ (cs'\ =\ \mathtt{interp}\ \mathtt{proc\_table}\ cs)$
\label{theorem:interp_correct}
\end{theorem}