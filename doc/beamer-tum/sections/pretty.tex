\section{Pretty Printing}

\begin{frame}[fragile]
\frametitle{Pretty Printing}
\framesubtitle{Factorial code generation}
\Fontvi

We take the following program in our semantics:

\begin{columns}[t]
\column{.45\textwidth}
\begin{semiverbatim}
definition factorial_decl :: fun_decl
  where "factorial_decl ==
    ( fun_decl.name = fact,
      fun_decl.params = [n],
      fun_decl.locals = [r, i],
      fun_decl.body =
        r ::= (Const 1);;
        i ::= (Const 1);;
        (WHILE (Less (V i) (Plus (V n) (Const 1))) DO
          (r ::= (Mult (V r) (V i));;
          i ::= (Plus (V i) (Const 1)))
        );;
        RETURN (V r)
    )"

definition main_decl :: fun_decl
  where "main_decl ==
    ( fun_decl.name = main,
      fun_decl.params = [],
      fun_decl.locals = [],
      fun_decl.body =
        n ::= Const 5;;
        r ::= fact ([V n])
    )"
\end{semiverbatim}
\column{.45\textwidth}
\begin{semiverbatim}
definition p :: program
  where "p ==
    ( program.name = fact,
      program.globals = [n, r],
      program.procs = [factorial_decl, main_decl]
    )"
\end{semiverbatim}

and export code for it.
\end{columns}


\end{frame}


\begin{frame}[fragile]
\frametitle{Pretty Printing}
\framesubtitle{Factorial code generation}
\Fontvi

We get the following code:

\begin{columns}[t]
\column{.65\textwidth}
\begin{semiverbatim}
\alert<2>{#include <stdlib.h>}
\alert<2>{#include <stdio.h>}
\alert<2>{#include <limits.h>}
\alert<2>{#include <stdint.h>}
\alert<2>{#include "../test_harness.h"}
\alert<2>{#include "../malloc_lib.h"}
\alert<3>{#ifndef INTPTR_MIN}
  \alert<3>{#error ("Macro INTPTR_MIN undefined")}
\alert<3>{#endif}
\alert<3>{#ifndef INTPTR_MAX}
  \alert<3>{#error ("Macro INTPTR_MAX undefined")}
\alert<3>{#endif}
\alert<3>{#if ( INTPTR_MIN + 1 != -9223372036854775807 )}
  \alert<3>{#error (" Assertion INTPTR_MIN + 1 == -9223372036854775807 failed")}
\alert<3>{#endif}
\alert<3>{#if ( INTPTR_MAX != 9223372036854775807 )}
  \alert<3>{#error (" Assertion INTPTR_MAX == 9223372036854775807 failed")}
\alert<3>{#endif}


\alert<4>{intptr_t n;}
\alert<4>{intptr_t r;}
\end{semiverbatim}
\column{.35\textwidth}
\begin{semiverbatim}
\alert<5>{intptr_t fact(intptr_t n) {}
\alert<5>{  intptr_t r;}
\alert<5>{  intptr_t i;}
\alert<5>{  r = (1);}
\alert<5>{  i = (1);}
\alert<5>{  while ((i) < ((n) + (1))) {}
\alert<5>{    r = ((r) * (i));}
\alert<5>{    i = ((i) + (1));}
\alert<5>{  }}
\alert<5>{  return(r);}
\alert<5>{}}

\alert<6>{intptr_t main() {}
\alert<6>{  n = (5);}
\alert<6>{  (r) = (fact(n));}
\alert<6>{}}
\end{semiverbatim}
\end{columns}


\end{frame}


\begin{frame}
\frametitle{Pretty Printing}
\framesubtitle{Casts}

We translate our programs into C's type intptr\_t type.
\bigskip

This type allows either a pointer or an integer to be stored in it.

All our variables are defined as a ``intptr\_t'' type.

\bigskip

Sometimes we want C to interpret our values as integers and other times as pointers.

We must be able to print casts to and from pointer values.

\bigskip
\pause

\begin{block}{Cast to pointer}
(intptr\_t *) $<$expression$>$

*((intptr\_t *) foo);
\end{block}

\begin{block}{Cast from pointer}
(intptr\_t) $<$expression$>$

(intptr\_t) \_\_MALLOC(sizeof(intptr\_t) * 5);
\end{block}


\end{frame}


\begin{frame}
\frametitle{Pretty Printing}
\framesubtitle{How do we translate malloc under our unlimited memory assumption?}

We translate malloc by wrapping it in another function.

\begin{block}{Code generation for malloc}
New (Const $9$) is pretty printed as \_\_MALLOC(sizeof(intptr\_t) * $9$)
\end{block}

\bigskip

This function will abort the program's execution upon an out of memory error.

\bigskip


\end{frame}


\begin{frame}
\frametitle{Exporting C code}
\framesubtitle{Exporting to a file}

We provide a mechanism that will allow to export the generated C program to a file in a given directory.


\end{frame}
